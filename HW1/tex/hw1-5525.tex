%!TEX program = pdflatex
%# -*- coding: utf-8 -*-
%!TEX encoding = UTF-8 Unicode

\documentclass[12pt,oneside,a4paper]{article}

%% ------------------------------------------------------
%% load packages
\usepackage{geometry}
\geometry{verbose,tmargin=2cm,bmargin=2cm,lmargin=2cm,rmargin=2cm}
\usepackage[pdfusetitle,
 bookmarks=true,bookmarksnumbered=true,bookmarksopen=true,bookmarksopenlevel=2,
 breaklinks=false,pdfborder={0 0 1},backref=false,colorlinks=false,
 unicode=true]
 {hyperref}
\hypersetup{pdfstartview={XYZ null null 1}}
\usepackage{url}
\setcounter{secnumdepth}{2}
\setcounter{tocdepth}{2}
\usepackage{microtype}

\usepackage{amsmath, amsthm, amssymb, amsfonts}
\usepackage[retainorgcmds]{IEEEtrantools}

% \usepackage{algorithm}
% \usepackage{algorithmic}
% \renewcommand{\algorithmicrequire}{\textbf{Input:}} 
% \renewcommand{\algorithmicensure}{\textbf{Output:}} 
\usepackage[algoruled, vlined]{algorithm2e}

\usepackage[T1]{fontenc}
\usepackage[utf8]{inputenc}
\usepackage[mono=false]{libertine}
\usepackage[libertine]{newtxmath}
\linespread{1.1}  
% \usepackage[toc,eqno,enum,bib,lineno]{tabfigures}

\usepackage{graphics}
\usepackage{graphicx}
\usepackage[figure]{hypcap}
\usepackage[hypcap]{caption}
\usepackage{tikz}
\usepackage{tikz-cd}
%\usepackage{grffile} 
%\usepackage{float} 
\usepackage{pdfpages}
\usepackage{pdflscape}
\usepackage{needspace}

\usepackage{multirow}
\usepackage{booktabs}
\usepackage{threeparttable}
\usepackage{dcolumn}
\usepackage{tabu}

\usepackage{verbatim}

\usepackage{etoolbox}
\BeforeBeginEnvironment{knitrout}{
    \begin{tabu} to \textwidth {XcX}
    \toprule[.7pt]
    & R Code Chunk & \\
    \bottomrule[.7pt]
    \end{tabu}
    \vspace*{-.5\baselineskip}
}
\AfterEndEnvironment{knitrout}{
    \vspace*{-.5\baselineskip}
    \noindent\rule{\textwidth}{.7pt}
}

%% Class, Exam, Date, etc.
\newcommand{\class}{STAT 5701: Statistical Computing}
\newcommand{\term}{Fall 2015}
\newcommand{\examnum}{Homework 1}
\newcommand{\hmwkTitle}{\class \\[1ex] \examnum}
\newcommand{\hmwkAuthorName}{Jingxiang Li}

\title{\hmwkTitle}
\author{\hmwkAuthorName}

\usepackage{fancyhdr}
\usepackage{extramarks}
\lhead{\hmwkAuthorName}
\chead{}
\rhead{\hmwkTitle}
\cfoot{\thepage}

\newcounter{problemCounter}
\newcounter{subproblemCounter}
\renewcommand{\thesubproblemCounter}{\alph{subproblemCounter}}
\newcommand{\problem}[0] {
    \clearpage
    \stepcounter{problemCounter}
    \setcounter{subproblemCounter}{0}
}

\newcommand{\subproblem}[0] {
    \stepcounter{subproblemCounter}
    \Needspace*{8\baselineskip}
    \vspace{1.8\baselineskip}
    \noindent{\textbf{\large{Problem \theproblemCounter.\hspace{1pt}\thesubproblemCounter}}}
    \vspace{\baselineskip}
    \newline
}

\newcommand{\solution} {
    \vspace{15pt}
    \noindent\ignorespaces\textbf{\large Solution}\par
}
\setlength\parindent{0pt}

%% some math shortcuts
\newcommand{\m}[1]{\texttt{{#1}}}
\newcommand{\E}[0]{\mathrm{E}}
\newcommand{\Var}[0]{\mathrm{Var}}
\newcommand{\sd}[0]{\mathrm{sd}}
\newcommand{\Cov}[0]{\mathrm{Cov}}
%%%%%%%%%%%%%%%%%%%%%%%%%%%%%%%%%%%%%%%%%%%%%%%%%%%%%%%%%%%%%%%%%%%%%%%%%%%

\begin{document}
\maketitle

\problem
\subproblem
The optimal $f(x)$ is $f(x) = E(y|x)$
\proof
The expected loss is
\begin{equation*}
\begin{aligned} 
E_{(x,y)}[l(f(x), y)] &= \int_{x}{\int_{y}{l(f(x), y)p(y|x)dy}p(x)}dx \\ 
&= \int_{x}{\int_{y}{(f(x) - y)^{2}p(y|x)dy}p(x)}dx
\end{aligned}
\end{equation*}
Here we minimize the expected loss
\begin{equation*}
\min_{f}\E_{(x,y)}[l(f(x), y)] \Leftrightarrow \min_{f}\int_{y}{(f(x) - y)^{2}p(y|x)dy}
\end{equation*}
To minimize the expected loss, we take the derivative with respect to $f$, and set it to 0.
\begin{equation*}
\begin{aligned} 
& \frac{\partial}{\partial f}\int_{y}{(f(x) - y)^{2}p(y|x)dy} = 0\\
\Rightarrow & \int_{y}{\frac{\partial}{\partial f}(f(x) - y)^{2}p(y|x)dy} = 0\\
\Rightarrow & \int_{y}{(f(x) - y)p(y|x)dy} = 0\\
\Rightarrow & \int_{y}{f(x)p(y|x)dy} = \int_{y}{yp(y|x)dy}\\
\Rightarrow & f(x) = E(y|x)
\end{aligned}
\end{equation*}
Q.E.D

\problem
\subproblem
\proof
The expected loss for $f$ is
\begin{equation*}
\begin{aligned} 
L(f) = p(f(x) \neq y) = \E_{(x,y)}[I(f(x) \neq y)] = \int_{x}{\sum_{y}{I(f(x) \neq y)p(y|x)}p(x)}dx
\end{aligned}
\end{equation*}
Here we minimize the expected loss with respect to $f$
\begin{equation*}
\begin{aligned}
&\min_{f}\int_{x}{\sum_{y}{I(f(x) \neq y)p(y|x)}p(x)}dx \\
\Leftrightarrow &\min_{f} \sum_{y}{I(f(x) \neq y)p(y|x)}\\
\Leftrightarrow &\min_{f} I(f(x) \neq 1)p(1|x) + I(f(x) \neq -1)p(0|x)
\end{aligned}
\end{equation*}
Notice that $p(0|x) > 0$, $p(1|x) > 0$ and $p(0|x) + p(1|x) = 1$.

Considering $I(f(x) \neq 1)$ and $I(f(x) \neq 0)$, there will be exactly one has value 1, and another one 0.

Hence once $p(1|x) > p(0|x)$, the optimal $f^*$ must satisfy $f^*(x) = 1$; otherwise $f^*(x) = -1$

i.e.
\begin{equation*}
f^*(x) = \left\{ \begin{array}{ll}
+1, & \mathrm{if} ~ p(1|x) > 1/2\\
-1, & \mathrm{otherwise}
\end{array}
\right.
\end{equation*}
Since $f^*$ is optimal, $L(f^*) \leq L(f)$

Q.E.D.
\end{document}